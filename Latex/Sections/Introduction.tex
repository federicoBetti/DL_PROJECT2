Lo scopo di questo progetto è realizzare un freamwork per la creazione e training di modelli di Artificial Neural Networks. \\
Come modello di partenza è stato utilizzato quello consigliato dall'assignements, tuttavia sono stati apportate modifiche e miglioramente durante lo sviluppo. La versione finale del framwork da la possibiltà di creare un multilayer perceptron con qualisasi grandezza e profondità. Abbiamo cercato di creare una struttura modulare grazie all'utilizzo del container Sequential per fornire la massima flessibiità al programmatore. Ogni livello di Sequential deve essere composto da un layer Dense e da una funzione di attivazione; la funzione linear può essere utilizzata nel caso in cui si voglia propagare l'output del Dense layer senza modifiche al livello sucessivo. \\
Le funzioni di attivazione implementate sono:
\begin{itemize}
	\item Linear
	\item Tanh
	\item Sigmoid
	\item Relu
	\item Softmax
\end{itemize}